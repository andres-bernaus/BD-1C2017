\section{Implementación  de Funcionalidades}

Una vez que terminado el modelo del problema, el siguiente paso fue comenzar a implementar la base de datos. Es decir generar en RethinkDB las distintas tablas, documentos y consultas necesarias para resolver el problema.\\

Para ello creamos un pequeño programa que genere de manera aleatoria todos los datos necesarios, asegurándose de ser consistente al momento de crear documentos para los enfrentamientos, campeonatos, escuelas y demás. Ya que, por ejemplo, si un competidor gana un enfrentamiento, el mismo debe ser contabilizado en el Documento $Campeonato$.\\ \\

Una vez llenadas las tablas con los suficientes datos, implementamos cada una de las consultas:\\ \\

\begin{itemize}

\item{\textbf{Cantidad de enfrentamientos ganados por competidor para un campeonato dado.}

Como esta consulta debe realizarse para un Campeonato determinado, el primer paso es filtrar la tabla $Campeonatos$ hasta obtener el documento indicado. Para ello utilizamos la función $filter$ sobre el campo $anio$ que, por lo visto anteriormente, identifica inequívocamente a cada Campeonato.\\

Una vez seleccionado el campeonato, hicimos un concatMap y un reduce sobre todos los pares $(competidor,enfrentamientosGanados)$ de todas las escuelas que participan del campeonato, obteniendo así los datos necesarios para esta consulta.

\begin{verbatim}

r.db('TP2').table('Campeonato').filter({anio: 2017})("escuelas").concatMap(function(elem){
  	return elem("competidores");
  })
  .reduce(function(left, right){
    return left.add(right);
  })

\end{verbatim}
}

\item{\textbf{Cantidad de medallas por nombre de escuela en toda la historia}

Usando la función map, podemos tomar cada documento $Escuela$ y devolver únicamente los campos que nos interesan para resolver la consulta, es decir, su $nombreEscuela$ y la sumatoria de todos los campos $cantidadMedallas$ que se encuentren en el arreglo $medallasPorCampeonato$:

\begin{verbatim}

r.db("TP2").table("Escuela").map(
function(elem){
  return {id: elem("idEscuela"),nombre: elem("nombreEscuela"),
    		cantidadTotal: elem("medallasPorCampeonato").sum("cantidadMedallas")};
	}
)

\end{verbatim}
}

\item{\textbf{para cada escuela, el campeonato donde gan\' m\'as medallas}


\end{itemize}