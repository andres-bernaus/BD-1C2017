\section{Consultas}
Una vez que terminado el modelo del problema, el siguiente paso fue comenzar a implementar la base de datos. Es decir generar en mySQL las distintas tablas, entradas y consultas necesarias para resolver el problema.\\

A continuación, mostraremos como se implementaron cada uno de los Store Procedures correspondientes a las consultas necesarias para cumplir con los requerimientos del punto 2 del enunciado:\\

\begin{itemize}
\item{ \textbf{Listado de inscritos en cada categoría.}\\
}

\item País que obtuvo mayor cantidad de medallas de oro, plata y bronce.
\item Sabiendo que las medallas de oro van 3 puntos, las de plata 2 y las de bronce 1 punto, se quiere realizar un ranking de puntaje por país y otro por escuela.
\item Dado un competidor obtener la lista de categorías donde haya participado y el resultado obtenido.

\item{ \textbf{Medallero por escuela.}\\
}
\item{\textbf{Listado de árbitros por país.}\\
CREATE PROCEDURE `arbitroPorPais`(in nombrePais varchar(45))\\
BEGIN\\
SELECT A.* FROM Arbitro A\\
	JOIN Pais P on A.idPais = P.idPais\\
    WHERE nombrePais = P.nombre;\\
END\\
}
\item La lista de todos los árbitros que actuaron como arbitro central en las modalidades de combate.

\item{\textbf{La lista de equipos por país.}\\
CREATE PROCEDURE `equiposPorPais`(in nombrePais varchar(45))\\
BEGIN\\
SELECT E.nombre FROM equipos E\\
	JOIN Competidor C ON C.idEquipo = E.idEquipo\\
	JOIN Escuela Esc ON Esc.idEscuela = C.idEscuela\\
	WHERE nombrePais = Esc.pais\\
END\\
}
\end{itemize}

