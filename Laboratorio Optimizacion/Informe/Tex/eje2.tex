\subsection{Ejercicio 2}\

Las consultas de este punto son las siguientes:

\begin{enumerate}[label=(\alph*)]
\item{SELECT \* FROM member WHERE region\_no $= 9$ AND $corp\_no = 126$}

\item{SELECT \* FROM member WHERE region\_no $= 9$ AND $corp\_no = 368$}

\end{enumerate}

Las probamos usando 2 índices distintos, uno sobre el campo corp\_no y el otro sobre region\_no.

\subsubsection{Non Clustered - corp\_no}

El plan de ejecución consta de un index seek para buscar los registros que cumplan la condición y un RID lookup para levantar las
filas correspondientes. Se unen los resultados con un nested loop join y se filtran para conseguir los que tengan número de región 9.
Los siguientes son los costos de esas operaciones. La siguiente tabla corresponde a la consulta a).

\begin{tabular}{| c | c | c | c | c |}
    \hline
    \text{Operación} & Costo Entrada/Salida & \text{Costo de CPU} & \text{Costo de Operador Estimado} & \text{Costo de Subarbol Estimado} \\ \hline
    Index seek & 0.003125 & 0.0001587 & 0.0032837(45\%) & 0.0032837 \\ \hline
    RID lookup & 0.003125 & 0.0001581 & 0.0040207(55\%) & 0.0040207 \\ \hline
	Nested Loops & 0 & 0.0000063 & 0.0000063(0\%) & 0.0073106 \\ \hline
	Filter & 0 & 0.0000007 & 0.0000007(0\%) & 0.0073113 \\ \hline
\end{tabular}

Los siguientes son los valores para la consulta b).

\begin{tabular}{| c | c | c | c | c |}
    \hline
    \text{Operación} & Costo Entrada/Salida & \text{Costo de CPU} & \text{Costo de Operador Estimado} & \text{Costo de Subarbol Estimado} \\ \hline
    Index seek & 0.003125 & 0.0001702 & 0.0032952(8\%) & 0.0032952 \\ \hline
    RID lookup & 0.003125 & 0.0001581 & 0.0354802(91\%) & 0.0354802 \\ \hline
	Nested Loops & 0 & 0.0000502 & 0.0000502(0\%) & 0.0388256 \\ \hline
	Filter & 0 & 0.0000058 & 0.0000057(0\%) & 0.0388313 \\ \hline
\end{tabular}

	
\subsubsection{Non Clustered - region\_no}

El plan de ejecución solo trata de un table scan para ambas consultas, los siguientes son los costos para los 2 scan:

\begin{tabular}{| c | c | c | c | c |}
    \hline
    \text{Operación} & Costo Entrada/Salida & \text{Costo de CPU} & \text{Costo de Operador Estimado} & \text{Costo de Subarbol Estimado} \\ \hline
    (a) & 0.160903 & 0.011157 & 0.17206(100\%) & 0.17206 \\ \hline
    (b) & 0.160903 & 0.011157 & 0.17206(100\%) & 0.17206 \\ \hline
\end{tabular}


\subsubsection{Conclusión}

La diferencia principal se da en la misma estructura del plan de ejecución. Mientras que las consultas realizadas con el índice sobre
corp\_no usan index seek para encontrar las filas que cumplan la condición pedida sobre corp\_no; las consultas realizadas con el
\'indice sobre region\_no, directamente realizaban un table scan sobre la totalidad de la tabla.

La explicación más probable involucra la cantidad de registros que coninciden con las condiciones pedidas. Hay 2 registros con corp\_no
igual a 126, 12 igual a 368 y hay más de 1100 con region\_no igual a 9. En los resultados del primer índice podemos ver que la parte más
costosa de la ejecución es el RID lookup. Pasar de 2 a 12 elementos aumentó el costo en un orden de magnitud. Cuando forzamos las consultas
a utilizar el segundo índice, el costo del lookup era 6 veces más alto que el total original. Probablemente por esta razón, el planificador
ignora el índice y realiza el scan.
