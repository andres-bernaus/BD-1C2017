\subsection{Ejercicio 2}\

A continuación ... siguientes consultas:
\begin{enumerate}[label=(\alph*)]
\item{SELECT \* FROM member WHERE region\_no $= 9$ AND $corp\_no = 126$}

\item{ SELECT \* FROM member WHERE region\_no $= 9$ AND $corp\_no = 368$}

\end{enumerate}


%--La idea aca era que el motor tiene que buscar entradas en la tabla que tenga corp=N y region=9. Las 2 consultas
%basicamente hacen y necesitan lo mismo. Lo unico interesante son los indices. Para el indice 2, uncluster por region, debe
%haber muchas filas de la tabla con region=9 por lo que no tiene sentido usar indices para despues tener que acceder a la
%tabla y corroborar que ademas tenga la corp buscada. ENtonces el motor ignora el indice 2  y directamente barre toda la
%tabla.
% Para el indice 1, en cambio, debe haber muchas menos filas con corp=N. Luego el motor decide levantar los indices, buscar
%aquellos indices que apuntan al N buscado;y una vez que tiene esas filas las filtra para quedarse con las de region=9



\subsubsection{Non Clustered - corp\_no}

Ejecutamos consultas y obtuvimos los siguientes resultados:
(esta tabla es copy paste de la anterior)
\begin{tabular}{| L | L | L | L | L |}
    \hline
    \text{Consulta} & CE/s & \text{C\_CPU} & \text{Operador Estimado} & \text{subArbol estimado} \\ \hline
    (a) & 0.165347 & 0.011157 & 0.176504 & 0.176504 \\ \hline
    (b) & 0.165347 & 0.0110785 & 0.176504 & 0.176504 \\ \hline   
\end{tabular}
	
\subsubsection{Non Clustered - region\_no}
			
\subsubsection{Sin indices}	
			
			


\subsubsection{Conclusión}

--EXPLICAR DIFERENCIAS, SI LAS HAY, ENTRE LAS 2--
