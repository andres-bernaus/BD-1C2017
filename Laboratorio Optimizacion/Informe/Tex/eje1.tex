\subsection{Ejercicio 1}

\subsubsection{SELECT * FROM member}
Para el primer consulta, notamos que los indices, tanto cluster como uncluster, no impactan en el tiempo de
ejecucion de la primer consulta. Esto es esperable, ya que la consulta  debe traer la tabla completa,
por que el uso del indice no tiene sentido. Eso se debe a que aumentaria el tiempo de ejecucion, sumandole el tiempo de  
acceso al indice.

\subsubsection{SELECT * FROM member WHERE member_no > 100}
Para la segunda consulta tenemos un escenario similar. Dado que gran parte de la tabla cumple la 
condicion member_no > 100, el uso del indice no impacta notablemente en la performance, dado 
que la cantidad de registros a traer de memoria es tan grande que predomina en el 
tiempo de ejecucion.

\subsubsection{SELECT member_no FROM member WHERE member_no > 100}
Para la tercer consulta, pudimos apreciar que usando el indice uncluster teniamos una mejora 
en el tiempo de ejecucion de un orden de magnitud. En principio nos parecio incorrecto, ya que
al buscar por datos mayores a una costante un indice cluster aparenta mas velocidad pues los datos 
se recorren lineamente. Sin embargo, luego de analizar la creacion de indices notamos que para el caso
uncluster, la tabla en que se almacena el indice contiene todos los datos que requerimos en esta consulta,
por lo que no necesitamos acceder a la base de datos, y sabemos (por lo explicado en el parrafo anterior) 
que para esta condicion el tiempo de traer los datos desde la base de datos es por mucho mayor que la ventaja
provista por el indice, ya sea cluster o uncluster.

\subsubsection{SELECT * FROM member WHERE member_no > 9990}
Para la cuarta consulta, podemos notar un variacion importante en los tiempos de ejecucion.
Tanto para el indice cluster como para el indice uncluster, en comparacion de la consulta
sin indice. Esto se debe a que ambos indices, se encuentran relacionados con la condicion 
de la consulta. Sin embargo, al comparar la consulta con el indice Cluster (llamemosla C), 
contra la consulta con indice uncluster ( llamemosla UNC), C es performante en comparacion 
de UNC. Esto se debe a que C, solo requiere hacer una busqueda por el indice. Una vez que 
llega al valor, solo necesita recorrer los registros siguientes pertinentes ya que estan 
ordenados. En el caso UNC, la consulta debe buscar cada valor de manera individual, pero al
ser solo un 0.001\% del total de la tabla, sigue siendo mucho mas performante que realizar 
un Table Scan.


\begin{comment}
\begin{enumerate}[label=(\alph*)]
\item{SELECT \* FROM member}

\item{SELECT \* FROM member WHERE member\_no $> 100$}

\item{SELECT member\_no FROM member WHERE member\_no $> 100$}

\item{SELECT FROM member WHERE member\_no $> 9990$}

\end{enumerate}

\subsubsection{Sin indices}	

Ejecutamos consultas y obtuvimos los siguientes resultados: \\

\begin{tabular}{| L | L | L | L | L |}
    \hline
    \text{Consulta} & CE/s & \text{C\_CPU} & \text{Operador Estimado} & \text{subArbol estimado} \\ \hline
	(a) & 0.165426 & 0.0110785 & 0.176504 & 0.176504 \\ \hline
	(b) & 0.165426 & 0.0110785 & 0.176504 &	0.176504 \\ \hline
	(c) & 0.165426 & 0.0110785 & 0.176504 &	0.176504 \\ \hline
	(d) & 0.165426 & 0.0110785 & 0.176504 &	0.176504 \\ \hline
\end{tabular}


\subsubsection{Non Clustered}

Ejecutamos consultas y obtuvimos los siguientes resultados:

\begin{tabular}{| L | L | L | L | L |}
    \hline
    \text{Consulta} & CE/s & \text{C\_CPU} & \text{Operador Estimado} & \text{subArbol estimado} \\ \hline
    (a) & 0.165347 & 0.011157 & 0.176504 & 0.176504 \\ \hline
    (b) & 0.165347 & 0.0110785 & 0.176504 & 0.176504 \\ \hline
    (c) & 0.0164583 & 0.011045 & 0.0275053 & 0.0275053 \\ \hline
    (d) & 0.003125 & 0.0001691 & 0.0032941 & 0.0032941 \\ \hline    
\end{tabular}
	
\subsubsection{Clustered}
			
\begin{tabular}{| L | L | L | L | L |}
    \hline
    \text{Consulta} & CE/s & \text{C\_CPU} & \text{Operador Estimado} & \text{subArbol estimado} \\ \hline
    (a) & 0.160903 & 0.011157 & 0.17206 & 0.17206 \\ \hline
    (b) & 0.159421 & 0.011047 & 0.170468 & 0.170468 \\ \hline
    (c) & 0.159421 & 0.011047 & 0.170468 & 0.170468 \\ \hline
    (d) & 0.003125 & 0.0001691 & 0.0032941 & 0.0032941 \\ \hline   
\end{tabular}	



\subsubsection{Conclusión}

--EXPLICAR DIFERENCIAS, SI LAS HAY, ENTRE LAS 2--
\end{comment}
