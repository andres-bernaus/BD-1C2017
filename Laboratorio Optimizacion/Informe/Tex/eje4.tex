\subsection{Ejercicio 4}

Para este ejercicio contamos con un único índice nonclustered sobre el campo $firstname$. El mismo es utilizado en ambas 
consultas para obtener las entradas de la tabla cuyo campo $firstname=UVI$. Con la diferencia que en la segunda consulta se 
utiliza las funciones $rtrim$ y $ltrim$ sobre el campo $firstname$:  
\begin{enumerate}[label=(\alph*)]
\item{SELECT \* FROM member WHERE firstname $=$ 'UVI'}

\item{SELECT \* FROM member WHERE rtrim(ltrim(firstname)) $=$ 'UVI'}

\end{enumerate}

Como ambas consultas son iguales (salvo por la condición del where), las ejecuciones sin indices arrojaron los mismos
resultados. Lo cual tiene sentido pues, necesariamente, se deberan recorrer la totalidad de la tabla $member$.\\

Analizaremos entonces las diferencias entre el comportamiento del motor de SQL al ejecutar ambas consultas, si utilizamos 
un índice nonclustered sobre el campo $firstname$.\\ 

\subsubsection{Non Clustered - member(firstname)}

Al ejecutar la consulta (a) obtuvimos el comportamiento esperado; el motor SQL aprobecho la existencia del indice y cargo 
aquellos valores que cumplían con la consulta. Obteniendo rápidamente las todas filas necesarias. \\%(?)

La consulta (b) en cambio, no utilizo el índice. Esto se debe a que dicha consulta ejecuta las funciones $ltrim$ y $rtrim$ 
sobre el campo $firstname$. Es decir que, se deben crear nuevas instancias de los valores del campo, y como resultado 
el motor no tiene la posibilidad de utilizarlos para agilizar la consulta.\\%chamuyo

Esto se puede observar en los resultados obtenidos como la consulta (a) se ejecuta varias magnitudes más rapido que la consulta (b):

%falta completar la tabla

\begin{tabular}{| l | l | L | L | L | L |}
    \hline
    Consulta & Operación & CE/s & \text{C\_CPU} & \text{Operador Estimado} & \text{subArbol estimado} \\ \hline
    (a) & RID Lookup & 0.003125 & 0.0001581 & 0.0032831(50\%) & 0.0032831 \\ \hline
    (a) & Index Seek & 0.003125 & 0.0001581 & 0.0032831(50\%) & 0.0032831 \\ \hline
    (a) & Inner Join & 0 & 0.0000042 & 0.0000042(0\%) & 0.0065704 \\ \hline
    (b) & Table Scan & 0.160903 & 0.011157 & 0.17206(99\%) & 0.17206 \\ \hline   
\end{tabular}

